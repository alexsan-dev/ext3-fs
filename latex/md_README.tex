Este proyecto contiene los archivos de entrada separados en carpetas. La carpeta {\ttfamily test} contiene archivos de entrada preliminares, para que puedan practicar antes de la calificación real. Cada día de calificación agregará una nueva carpeta para los archivos de ese día. Las entradas preliminares solo validan mínimos. Por lo que la dificultad es un tanto reducida en comparación con los reales. Sin embargo, son un buen punto de entrada para validar mínimos y el correcto funcionamiento del proyecto. ~\newline
 Para correr los archivos de entrada, seguir las intrucciones que se detallan en este documento.

\subsection*{Requerimientos}

El proyecto deberá ser presentado desde un entorno válido. Las opciones aceptadas son las siguientes.
\begin{DoxyItemize}
\item Instalación física Linux (la distribución queda a discreción del estudiante).
\item Entorno Mac.
\item Instalación $\ast$\+B\+SD física (la distribución queda a discreción del estudiante).
\item W\+SL. En caso de utilizar esta opción, se les recuerda que deben preparar T\+O\+DO su entorno de acuerdo al documento W\+S\+L.\+pdf adjunto en este repositorio a modo de recordatorio.
\end{DoxyItemize}

\subsection*{Utilización}

Una vez resuelto el problema del entorno, el proyecto deberá ser compilado y ejecutado desde consola. En ningún momento se aceptará su ejecución desde un I\+DE o algún otro método auxiliar. El nombre del ejecutable queda a discreción del estudiante. Una vez compilado seguir los siguientes pasos para su ejecución
\begin{DoxyItemize}
\item Instalar Zsh. Zsh es una shell, es incluída por defecto en algunas distribución Linux, así como en Mac\+OS. Puedes verificar si tienes instalado zsh con el comando {\ttfamily zsh -\/-\/version}. Si tu sistema no la incluye por defecto, deberás instalarla. NO es necesario que cambies la shell por defecto de tu usuario, únicamente con instalar zsh como una shell secundaria es suficiente.
\item Editar el archivo {\ttfamily env.\+zsh}; Este archivo define algunas variables de entorno que necesitarán para correr el archivo de entrada. Estas variables están acompañadas de un comentario para mejor comprensión
\item {\ttfamily zsh no-\/exec.\+zsh}
\item {\ttfamily \$\+E\+J\+E\+C\+U\+T\+A\+B\+LE exec -\/path=exec.\+sh}
\item Repetir el último comando según archivos $\ast$.sh existan. El orden será indicado por el auxiliar. No intentes ejecutar archivos $\ast$.sh con exec antes de ejecutar los archivos no-\/exec!
\end{DoxyItemize}

\subsection*{Consideraciones}


\begin{DoxyItemize}
\item Los archivos $\ast$.sh (a ser ejecutados bajo el comando exec) NO contienen errores, a menos que se indique explícitamente lo contrario. Es decir, hay archivos $\ast$.sh diseñados especialmente para contener errores y evaluar el manejo de los mismos y la robustez de su aplicación. Sin embargo, estos N\+U\+N\+CA se van a mezclar con archivos $\ast$.sh normales, en el caso de los archivos de prueba donde solamente se evalúan requerimientos mínimos estos no contienen errores.
\item El comando mount NO genera I\+Ds de manera aleatoria. Los primeros dígitos son los últimos dígitos de su carnet. Seguidos de un número, comenzando en 1 e incrementando de acuerdo al orden que se montan distintos discos. Seguido de una letra en minuscula, comenzando en a e incrementando según el orden de particiones montadas en un mismo disco. Esto significa que el nombre esperado de las particiones deberá ser predecible y coincidir con el esperado en los archivos de calificación (cada comando mount va acompañado de un comentario que indica el ID esperado).
\item No deben preocuparse por crear discos o reportes en locaciones inválidas según un sistema Unix. La mayoría de operaciones se realizarán en el directorio /tmp.
\end{DoxyItemize}

\subsection*{Conceptos a evaluar}

El proyecto puede ser dividido en 4 partes que serán evaluadas y constituyen el 100\% del proyecto.
\begin{DoxyEnumerate}
\item Implementacioń del esquema de partición M\+BR.
\item Implementación del sistema de archivos ext2/3.
\item Implementación de comandos de utilidad para el manejo del sistema de archivos (cp, mv, rm, touch, etc).
\item Implementación del esquema de grupos, permisos y journaling. Los reportes no representan ninguna parte del proyecto ya que son obligatorios para demostrar la correcta implementación de estos conceptos y se utilizan en todas las partes. No todos los reportes son obligatorios ya que no todas las partes son requerimientos mínimos. Sin embargo, para tener derecho a la ponderación de cada una de estas partes deberán ser presentados los reportes según aplique (i.\+e, para tener derecho a los puntos por implementar el sistema de archivos ext2/3, deberá ser presentado el reporte tree).
\end{DoxyEnumerate}

\subsection*{Requerimientos mínimos}

Los requerimientos mínimos representan el 25\% del proyecto. Estos incluyen únicamente el punto 1. Implementación del esquema de partición M\+BR. A modo de recordatorio se listan los comandos y funcionalidades que pertenecen a esta área.
\begin{DoxyItemize}
\item Aplicación por argumentos de consola (no exec).
\item Comando exec.
\item mkdisk
\item fdisk
\item rmdisk
\item mount
\item umount
\item rep -\/name=mbr
\item rep -\/name=disk
\item Dentro del comando fdisk se implica la implementación de particiones Primarias, Extendidas y Lógicas.
\end{DoxyItemize}

\subsection*{Notas}


\begin{DoxyItemize}
\item Los argumentos de consola se utilizan únicamente para mkdisk, fdisk, rmdisk. mount, umount, rep, etc. Son utilizados únicamente desde archivos exec.
\item En los requerimientos mínimos se lista como último punto la implementación de particiones Primarias, Extendidas y Lógicas, a modo hacer explícito su requerimiento como parte del comando fdisk.
\item Los archivos de calificación varían de acuerdo al día.
\item La hoja de calificación será subida en este mismo repositorio.
\item El ejecutable {\ttfamily proyecto1} que se encuentra en este repositorio no hace nada. Es solamente un placeholder.
\item La carpeta {\ttfamily data} es una carpeta con información de prueba que será utilizada para llenar los sistemas de archivos durante la calificación. Su contenido no tiene ninguna trascendencia o relevancia más de allá de este propósito.
\item El alumno voluntario durante la explicación del entorno W\+SL puede hacer uso del mismo sin necesidad de configurar todo su entorno según las instrucciones detalladas en W\+S\+L.\+pdf; Sin embargo, la instalación de zsh para la ejecución de los archivos no-\/exec es necesaria. 
\end{DoxyItemize}